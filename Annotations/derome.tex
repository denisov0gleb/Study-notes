	\documentclass[12pt, a4paper, oneside]{article}	% {article|letter|report}

	\usepackage[utf8]{inputenc}
	\usepackage[T1, T2A]{fontenc}
	\usepackage[english, russian]{babel}
	\usepackage[warn]{mathtext}
	\usepackage{amsmath}
	%\usepackage{amsfonts}
	\usepackage{amssymb}

	\usepackage{graphicx}

	\graphicspath{ {/}}
	%\usepackage[version=4]{mhchem}
	\usepackage{enumerate}
	\usepackage{fancybox, fancyhdr}
	\fancyheadoffset[R]{0.05cm} %так можно регулировать ширину колонтитула


	\usepackage[left=25mm, top=20mm, right=20mm, bottom=20mm]{geometry}
	%\textheight=28cm			% высота текста A4
	%\textwidth=18cm				% ширина текста A4
	%\oddsidemargin=0pt			% отступ от левого края
	%\topmargin=0cm			% отступ от верхнего края
	\parindent=0pt				% абзацный отступ
	\parskip=0pt				% интервал между абзацами
	\tolerance=2000				% терпимость к "жидким" строкам
	\flushbottom				% выравнивание высоты страниц
	\setcounter{secnumdepth}{0}	% секции без нумерации

\begin{document}
	\subsection
		{Почему ЯМР с разверткой поля такое медленное?}

		Пусть максимально ожидаемое разрешение между пиками составляет 1 Гц.

		Тогда из уравнения 
		$$E = h \nu$$
		получаем, что $\delta E = h$.

		Из принципа неопределённости Гейзенберга имеем:

		$$\delta E \cdot \delta t ~ h$$

		Следовательно, максимальная скорость съёмки для ЯМР с развёрткой поля составляет 1 Гц{\slash}с.

		При съёмке спектра шириной в 1000 Гц (например 10 м.~д. на приборе с частотой 100 МГц)
		потребуется около 15 минут.

	\subsection {Как один импульс возбуждает сразу несколько частот?}
		
		Принцип неопределённости Гейзенберга:

		$$E \cdot \delta t = h \cdot \delta \nu \cdot \delta t ~ h$$

		Тогда имеем:

		$$\delta \nu \cdot \delta t ~ 1$$

		Тогда для монохроматического источника излучения появляется неопределённость на частоте
		$\frac{1}{\delta t}$ Гц при длительности импульса $\delta t$ с.

		Поэтому для возбуждения всех сигналов в спектре шириной 1000 Гц требуется импульс
		длительностью около 1 мс (на самом деле меньше вследствие ряда дополнительных причин).

	\subsection {Как понять, сколько точек необходимо для оцифровки?}

		Пусть требуется разрешение пиков до $0{,}2$ Гц при съёмке на приборе с рабочей частотой
		500~МГц на ширине 10 м.~д. (5000 Гц).

		Применим критерий Найквиста:

		Если есть частоты $N$ Гц, то выборка сигнала должна проводить каждые $\frac{1}{2N}$ с.

		Тогда при длительности записи на 5 м.~д. 5 секунд с частотой выборки 10000 Гц $TD = 50k$.

	\subsection {Отражённые пики}

	Отражённые{\slash}мнимые пики --- это пики с большей частотой, чем несущая, вследствие чего
	часть информации о них в ходе Фурье преобразования теряется.

	Чтобы понять является ли пик отражённым или настоящим, надо поменять центр спектра (несущую
	частоту). Реальные пики будут иметь те же значения химических сдвигов, а отражённые --- поменяют.

	\subsection {Откуда берётся шум в спектре?}

		\begin{enumerate}
			\item
				Магнитная восприимчивость материалов датчика не нулевая

			\item
				Недостаточная однородность радиочастотного поля $B_1$: имеется коэффициент для
				приёма{\slash}передачи радиочастотного импульса датчика.

			\item
				Сам образец создаёт неоднородности поля за счёт наличия границ фаз.
		\end{enumerate}
\end{document}
